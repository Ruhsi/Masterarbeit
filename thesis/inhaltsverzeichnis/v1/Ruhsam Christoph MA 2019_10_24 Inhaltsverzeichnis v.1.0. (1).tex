%%% File encoding: UTF-8
%% äöüÄÖÜß  <-- no German Umlauts here? Use an UTF-8 compatible editor!

%%% Magic comments for setting the correct parameters in compatible IDEs
% !TeX encoding = utf8
% !TeX program = pdflatex 
% !TeX spellcheck = de_DE
% !BIB program = biber

\documentclass[master,german]{hgbthesis}
% Permissible options in [..]: 
%   Type of work: diploma, master (default), bachelor, internship 
%   Main language: german, english (default)
%%%----------------------------------------------------------

\RequirePackage[utf8]{inputenc}		% Remove when using lualatex or xelatex entfernen!
\usepackage{graphicx}
\usepackage{svg}
\usepackage{minted}
\usepackage{listings}
\usepackage{varioref}
\usepackage{longtable}
\usepackage{tabularx}
\usepackage{parskip}
\usepackage{url}
\geometry{top=2.5cm,left=2.5cm,bottom=2.5cm,right=2.5cm}

% -----------------------------------------------------
\newenvironment{code}{\captionsetup{type=listing}}{}
% -----------------------------------------------------
% -----------------------------------------------------
\newcommand{\mysubsubsection}[1]{{\subsubsection{\textbf{#1}}}}
\newcommand{\mentionedtext}[1]{{\textit{{#1}}}}
\newcommand{\sourceDir}{./sources}
\newcommand{\sourceFontSize}{\fontsize{10pt}{11.5}}
\newcommand{\quotes}[1]{``#1''}
\newmintedfile[bashFile]{bash}{
	linenos=false, 
	frame=none, 
	breaklines=true, 
	tabsize=2,
	numbersep=5pt,
	xleftmargin=10pt,
	baselinestretch=1,
	fontsize=\sourceFontSize
}
\newmintedfile[yamlFile]{yaml}{
	linenos=false, 
	frame=none, 
	breaklines=true, 
	tabsize=2,
	numbersep=5pt,
	xleftmargin=10pt,
	baselinestretch=1,
	fontsize=\sourceFontSize
}
\newmintedfile[javaFile]{java}{
	linenos=false, 
	frame=none, 
	breaklines=true, 
	tabsize=2,
	numbersep=5pt,
	xleftmargin=10pt,
	baselinestretch=1,
	fontsize=\sourceFontSize
}
\newmintedfile[xmlFile]{xml}{
	breaklines=true, 
	tabsize=2,
	numbersep=5pt,
	xleftmargin=10pt,
	baselinestretch=1,
	autogobble=true,
	breakautoindent=false,
	fontsize=\sourceFontSize
}
\newmintinline[inlineJava]{java}{
	fontsize=\sourceFontSize
}
\newmintinline[inlineBash]{bash}{
	fontsize=\sourceFontSize
}
% -----------------------------------------------------
\graphicspath{{images/}}    % location of images and graphics
\logofile{logo}				% logo file = images/logo.pdf (use \logofile{} for no logo)
\bibliography{references.bib}  	% name of bibliography file (references.bib)
\setlength{\parindent}{0pt}
\usepackage{titlesec}

\titleformat{\section}
{\normalfont\Large\bfseries}{\thesection}{1em}{}

\renewcommand*{\labelalphaothers}{}
\DeclareLabelalphaTemplate{
	\labelelement{
		\field[final]{shorthand}
		\field{labelname}
		\field{label}
	}
	\labelelement{
		\literal{,\addhighpenspace}
	}
	\labelelement{
		\field{year}
	}
}


%%%----------------------------------------------------------
% Title page entries
%%%----------------------------------------------------------

%%% Entries for ALL types of work: --------------------------
%\title{Implementierung einer Partnerdatenbank und Evaluierung des verwendeten Technologiestack} %no camel used (and Camel)
%\author{Ing. Thomas Herzog B.Sc}
%\programname{Software Engineering}
\placeofstudy{Hagenberg}
\dateofsubmission{2020}{05}{30}	% {YYYY}{MM}{DD}

%%% Entries for Bachelor theses only: -----------------------
%\thesisnumber{XXXXXXXXXX-A}   %e.g. 1310238045-A  
% (Stud-ID, A = 1st Bachelor thesis)
%\semester{Fall Semester 2017} 	% Fall/Spring Semester YYYY
%\coursetitle{Introduction to Trivial Problems 1} 
\advisor{FH-Prof. DI Dr. Herwig Mayr}

%%% Restricted publication license instead of CC (master only):
%\cclicense

%%%----------------------------------------------------------
\begin{document}
%%%----------------------------------------------------------

%%%----------------------------------------------------------
\frontmatter							% title part (roman page numbers)
%%%----------------------------------------------------------

\includepdf{title.pdf}
%\maketitle
\tableofcontents

%\include{front/preface} 	% preface is optional
%\chapter{Abstract}
This thesis demonstrates the implementation of a microservice architecture. The deployment of the microservice architecture in OpenShift is also shown.

Microservices are an architectural approach to break down large, heavy-weight applications into small, consistent, and clearly delimited services. At the beginning, microservices are differentiated from service-oriented architectures and the advantages and disadvantages of microservice architectures are discussed.

For the deployment of microservices in OpenShift, the microservices must be containerized with Docker beforehand. The process of containerization and Docker are also dealt with in this thesis.

OpenShift is based on Kubernetes. Kubernetes is a container-as-a-service platform for managing and orchestrating containers. The components and the main elements of Kubernetes will be covered in this thesis, since they are relevant for the use of OpenShift.

However, container-as-a-service is not enough for people without an understanding of containerization and orchestration. Platform-as-a-service is required for this. OpenShift is a platform-as-a-service platform and offers not only a container runtime, but also tools for building, deploying, and monitoring containerized applications. OpenShift also provides security mechanisms for securing the applications. In this thesis the main components of OpenShift and the OpenShift objects are described in more detail.

The main part of this thesis is the design and implementation of the partner database. The partner database is an application for the 3 Banken IT GmbH and is used to create, delete, and display companies and partners. The partner database is a prototype for the 3 Banken IT GmbH to create a microservice architecture and deploy the services in OpenShift. Microservice tools for easier development are also presented in this thesis.

This is followed by an evaluation of the application, the description of the tests and a summary of this thesis. A brief outlook on the further development of the partner database forms the conclusion.


%%%----------------------------------------------------------
\mainmatter          			% main part (arabic page numbers)
%%%----------------------------------------------------------

\chapter{Einleitung ... 5.5 Seiten}
\section{Motivation zur Architektur von Microservices ... 1.5 Seiten}
\section{Motivation zum Einsatz von Cloudtechnologien ... 1.5 Seiten}
\section{Zielsetzung der Implementierung der Partnerdatenbank ... 1 Seite}
\section{Ziel des Deployment der Partnerdatenbank in OpenShift ... 1 Seite}
\section{Leitfaden und Gliederung der Schrift ... 0.5 Seiten}

\chapter{Serviceorientierte Architektur und Microservices ... 12.5 Seiten}
\section{Definition und Abgrenzung ... 4 Seiten}
\section{Vergleich zu monolithischer Architektur ... 2 Seiten}
\section{Charakteristiken ... 3 Seiten}
\section{Varianten ... 1.5 Seiten}
\section{Vor- und Nachteile ... 2 Seiten}


\chapter{Containerisierung mit Docker ... 6 Seiten}
\section{Docker ... 4 Seiten}
\section{Notwendigkeit von Containerisierung ... 2 Seiten}


\chapter{OpenShift ... 10.5 Seiten}
\section{Beschreibung von OpenShift ... 2 Seiten}
\section{Komponenten von Kubernetes ... 3.5 Seiten}
\section{Die OpenShift-Umgebung ... 3 Seiten}
\section{Fabric8 ... 2 Seiten}

\chapter{Partnerdatenbank ... 14.5 Seiten}
\section{Grundaufbau und Ziel der Partnerdatenbank ... 2.5 Seiten}
\section{Backend-Beschreibung ... 4 Seiten}
\section{Beschreibung der einzelnen Services ... 4 Seiten}
\section{Frontend-Beschreibung ... 4 Seiten}

\chapter{Design der Partnerdatenbank ... 11 Seiten}
\section{Microservice-Architektur ... 3 Seiten}
\section{Beschreibung der verwendeten Microservice-Technologien ... 6 Seiten}
\section{Design in OpenShift .. 2 Seiten}


\chapter{Implementierung der Partnerdatenbank ... 17 Seiten}
\section{Microservice-Architektur ... 2 Seiten}
\section{Automatisierte Test-, Build- und Deployment-Pipelines mit Jenkins ... 3 Seiten}
\section{Fehlerbehandlung mit Microprofile ... 2 Seiten}
\section{REST-Schnittstellenbeschreibung mit Swagger ... 2 Seiten}
\section{Tracing mit Jaeger ... 1 Seite}
\section{Einsatz von Docker zur Containerisierung der Anwendung ... 1 Seite}
\section{Konfiguration und Deployment-Deskriptoren von OpenShift ... 4 Seiten}
\section{Deployment in OpenShift mit Fabric8 ... 2 Seiten}


\chapter{Evaluierung der Anwendung ... 5.5 Seiten}
\section{Evaluierung des Frontend ... 1.5 Seiten}
\section{Unittests ... 1 Seite}
\section{Integrationstests ... 1 Seite}
\section{Architekturevaluierung ... 2 Seiten}

\chapter{Zusammenfassung und Ausblick ... 2-3 Seiten}

\chapter{Quellenverzeichnis}



%%%----------------------------------------------------------
\end{document}
%%%----------------------------------------------------------