\chapter{Abstract}
An Enterprise Service Bus (ESB) is a crucial part of an enterprise, which connects the enterprise to its partners, customers, and other branches. The appearance of containerization, cloud services, and the microservice architecture have provided new possibilities for implementing and running an ESB. But, an ESB is commonly used by large conservative enterprises, which don't adapt new technologies fast, and wait until a new technology has proven itself. Especially the cloud is something the industry denied to use for a long time, because of the fact, that the infrastructure and data are managed and maintained by external service providers. \\ 

These days, we live in the so called cloud age, whereby global enterprises like Red Hat or Amazon provide cloud services such as Platform as a Service (PaaS), which can scale with the business. Enterprises start to consider to move their ESB installations to the cloud to profit from the cloud service provided features. Moving an ESB to the cloud will be a long term process for an enterprise, because the established processes for development, running, and managing the ESB will have to change. \\

This thesis has the goal to give the reader an overview of the cloud related concepts and technologies such as, Infrastructure as Code (IaC) and Docker, which are the base for cloud services. The implemented ESB prototype,  is  available at \url{https://github.com/cchet-thesis-msc/prototype}, and shows how an ESB could be implemented on a PaaS platform. \\


