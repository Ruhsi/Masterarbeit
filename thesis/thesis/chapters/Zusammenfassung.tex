\chapter{Zusammenfassung}
In diesem Kapitel erfolgt eine Zusammenfassung der Arbeit. Auch ein Ausblick über die weitere Entwicklung der Partnerdatenbank wird gegeben.

\section{Resümee}
Das Erstellen einer Microservice-Architektur ist sehr komplex. Es muss dabei viel Zeit in Planung und Designentscheidungen investiert werden. Welche Designvariante wird gewählt? Gibt es ein zentrales Service, über das alle Geschäftsfälle abgewickelt werden, oder sollen alle Services miteinander kommunizieren können? Diese Entscheidung muss vor der Implementierung gründlich durchdacht werden, da sie die Implementierung stark beeinflusst.
Microservice-Architekturen bieten eine Vielzahl an Vorteilen gegenüber Monolithen. Dazu zählen die Technologieunabhängigkeit, einfache Skalierbarkeit und die Austauschbarkeit von einzelnen Microservices. Es sollten jedoch auch die Nachteile bedacht werden. Microservices laufen nicht in derselben JVM und REST-Aufrufe haben höhere Latenzzeiten als einfache Methodenaufrufe. Jedes Microservice muss einzeln deployt werden. Dazu muss in jedem Microservice ein eigener Deployment-Prozess angelegt werden. Dies bedeutet weiteren Aufwand für Entwickler. Microservice-Architekturen sind keine Allheilmittel für Designentscheidungen. Auch Monolithen haben viele Vorteile. Es muss vor der Implementierung einer Anwendung gründlich überlegt werden, ob der Einsatz einer Microservice-Architektur Sinn macht.

Der Verfasser dieser Arbeit hat sich am Beispiel der Partnerdatenbank für die Designvariante Service-Orchestrierung entschieden. Mit dem Partner-Service liegt ein zentrales Service vor, mit dem alle Geschäftsfälle abgewickelt werden. Auch bezüglich Sicherheitskonfiguration ist die Service-Orchestrierung einfacher zu designen und zu entwickeln. Es muss in diesem Prototyp lediglich das Partner-Service abgesichert werden, da nur dieses nach außen verfügbar ist und mit den weiteren Services intern kommuniziert.

OpenShift als Platform-as-a-Service-Plattform wurde von der 3 Banken IT GmbH vorgegeben, da das Unternehmen bereits andere Applikationen in OpenShift laufen lässt. OpenShift bietet dabei eine Palette an Tools zum Erstellen und Managen von Applikationen in der Cloud. Es ist zudem Open Source und kann kostenlos genutzt und selbst gehostet werden. Auch Sicherheitsmechanismen, wie Namespaces, werden von OpenShift gemanagt und bieten dem Entwickler weitere Abschirmung der Services.
OpenShift bietet ebenfalls eine Integration von Continous Integration und Continous Deployment an. Dadurch werden automatische Builds und Deployments von Services erlaubt. Plattformen wie OpenShift werden deshalb oft in den Entwicklungsprozess mit eingebunden. Dies verringert den Aufwand von Entwicklern und bietet zusätzliche Automatisierung.

Die Partnerdatenbank dient als Prototyp für die 3 Banken IT GmbH und soll als Demonstration zur Entwicklung einer Microservice-Architektur und dem Deployment in OpenShift dienen. Spring Boot bietet dabei eine Fülle von Abhängigkeiten zur erleichterten Entwicklung von Microservices an. Auch das Deployment in OpenShift gestaltet sich mit dem fabric8-maven-plugin sehr einfach. Weiters wurden auch Tools für die Beschreibung von REST-Schnittstellen und dem Tracing der Serviceaufrufe vorgestellt.

Das Ergebnis dieser Arbeit ist lediglich ein Prototyp. Dabei mangelt es noch an einer Fehlerbehandlung auf Backend- und Frontend-Seite. Auch das Design des Frontends wurde mit der 3 Banken IT GmbH noch nicht genauer geklärt, da diese Applikation in eine bestehende Portallösung integriert und das einheitliche Design behalten werden soll. Deshalb ist auch das Design des Frontends auf ein Minimum beschränkt. 

\section{Ausblick}
Zukünftig sollen in dieser Applikation nicht nur die Geschäftsfälle \textit{Unternehmen anzeigen}, \textit{Unternehmen anlegen}, \textit{Unternehmen deaktivieren}, \textit{Partner anzeigen}, \textit{Partner anlegen}, \textit{Partner deaktivieren} integriert werden. Auch weitere Geschäftsfälle, wie die Angebotslegung oder die Kündigung, sollen eingebunden werden.

Durch die Microservice-Architektur wurde die beste Voraussetzung zur Integration weiterer Services geboten. Auch das Deployment neuer Services in OpenShift gestaltet sich durch den bestehenden Code sehr einfach. Dabei soll in Zukunft auch das Paradigma Infrastructure-as-Code verfolgt werden. Die gesamte Infrastruktur soll als Code vorliegen und es sollte so wenig wie möglich manuell in der Web Console konfiguriert werden.

Der Verfasser der Arbeit weiß nicht, ob die bestehende Portallösung, in die diese Applikation integriert werden soll, bereits in OpenShift läuft. Falls ja, sollte die Integration für die 3 Banken IT GmbH kein Problem darstellen, da das Deployment der Datenbank, der Backend-Services und des Frontends in dieser Arbeit genau beschrieben wurden.

Dabei müssen jedoch von der 3 Banken IT GmbH weitere Sicherheitsmechanismen integriert werden. Dazu zählen die Fehlerbehandlung auf Backend- und Frontend-Seite, sowie die richtige Konfiguration von CORS, sobald die URL des Frontends bekannt ist.
Gerade bei der Entwicklung von Software für Banken muss großer Wert auf Sicherheitsmechanismen gelegt werden.

Das Deployment der Partnerdatenbank wurde für die OpenShift-Version 3 gezeigt, da diese von der 3 Banken IT GmbH vorgegeben wurde. Zum Zeitpunkt dieser Arbeit war jedoch bereits die OpenShift-Version 4 verfügbar. RedHat hat in dieser Version einige Komponenten verändert. Das Deployment, wie in dieser Arbeit beschrieben, wurde für die OpenShift-Version 4 nicht getestet. Der Verfasser gibt keine Garantie, dass die Partnerdatenbank auch in OpenShift 4 läuft.