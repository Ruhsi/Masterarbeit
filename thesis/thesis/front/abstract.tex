\chapter{Abstract}
This thesis demonstrates the implementation of a microservice architecture. The deployment of the microservice architecture in OpenShift is also shown.

Microservices are an architectural approach to break down large, heavy-weight applications into small, consistent, and clearly delimited services. At the beginning, microservices are differentiated from service-oriented architectures and the advantages and disadvantages of microservice architectures are discussed.

For the deployment of microservices in OpenShift, the microservices must be containerized with Docker beforehand. The process of containerization and Docker are also dealt with in this thesis.

OpenShift is based on Kubernetes. Kubernetes is a container-as-a-service platform for managing and orchestrating containers. The components and the main elements of Kubernetes will be covered in this thesis, since they are relevant for the use of OpenShift.

However, container-as-a-service is not enough for people without an understanding of containerization and orchestration. Platform-as-a-service is required for this. OpenShift is a platform-as-a-service platform and offers not only a container runtime, but also tools for building, deploying, and monitoring containerized applications. OpenShift also provides security mechanisms for securing the applications. In this thesis the main components of OpenShift and the OpenShift objects are described in more detail.

The main part of this thesis is the design and implementation of the partner database. The partner database is an application for the 3 Banken IT GmbH and is used to create, delete, and display companies and partners. The partner database is a prototype for the 3 Banken IT GmbH to create a microservice architecture and deploy the services in OpenShift. Microservice tools for easier development are also presented in this thesis.

This is followed by an evaluation of the application, the description of the tests and a summary of this thesis. A brief outlook on the further development of the partner database forms the conclusion.
