\chapter{Kurzfassung}
In dieser Arbeit wird die Implementierung einer Microservice-Architektur von Grund auf gezeigt. Auch das Deployment dieser Microservice-Architektur in OpenShift wird gezeigt. 

Microservices sind ein Architekturansatz, um große, schwergewichtige Applikationen in kleine, konsistente und voneinander klar abgegrenzte Services zu zerlegen.
Zu Beginn werden Microservices von service-orientierten Architekturen abgegrenzt und die Vor- und Nachteile von Microservice-Architekturen behandelt.

Für das Deployment der Microservices in OpenShift müssen die Microservices vorher mit Docker containerisiert werden. Auch der Prozess der Containerisierung und Docker werden in dieser Arbeit behandelt. 

OpenShift setzt auf Kubernetes auf. Kubernetes ist eine Container-as-a-Service-Plattform zum Managen und Orchestrieren von Containern. Die Komponenten und Hauptbestandteile von Kubernetes werden ebenfalls in dieser Arbeit behandelt, da diese relevant für den Einsatz von OpenShift sind.

Container-as-a-Service reicht jedoch für Menschen ohne Verständnis für Containerisierung und Orchestrierung von Containern nicht aus. Dafür wird Platform-as-a-Service benötigt. OpenShift ist eine Platform-as-a-Service-Plattform und bietet nicht nur eine Container Runtime, sondern auch Tools zum Bauen, Deployen und Monitoring von containerisierten Applikationen. Auch Sicherheitsmechanismen zum Absichern der Anwendungen stellt OpenShift bereit.
In dieser Arbeit werden die Hauptbestandteile von OpenShift und die OpenShift-Objekte näher beschrieben.

Den Hauptteil dieser Schrift macht das Design und die Implementierung der Partnerdatenbank aus. Die Partnerdatenbank ist eine Applikation für die 3 Banken IT GmbH und dient zum Erstellen, Löschen und Anzeigen von Unternehmen und Partnern. Die Partnerdatenbank ist ein Prototyp für die 3 Banken IT GmbH zur Erstellung von Microservice-Architekturen und dem Deployment in OpenShift. Auch Microservice-Tools zur einfacheren Entwicklung werden in dieser Arbeit vorgestellt.

Abschließend folgt eine Evaluierung der Anwendung sowie die Beschreibung der Tests und eine Zusammenfassung über diese Schrift. Auch ein kurzer Ausblick über die weitere Entwicklung der Partnerdatenbank wird gegeben.